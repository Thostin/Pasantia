\documentclass[11pt, a4paper]{scrartcl}

%  No sé por qué quieren que la fuente sea Arial. 
%  Helvetica es lo que más se aproxima con pdflatex
\usepackage{helvet}
\renewcommand{\familydefault}{\sfdefault}

\usepackage[sexy]{evan}           % loads its own layout
\usepackage[
  paperwidth=216mm,
  paperheight=330mm,
  top=1in
]{geometry}

\setlength{\headheight}{20pt}   % change to desired value
\setlength{\headsep}{22pt}      % optional: space between header and text
\setlength{\topmargin}{0pt}     % optional: top margin adjustment

\usepackage{multirow}

\usepackage[spanish, es-nosectiondot, es-nolists, shorthands=off]{babel}
\begin{document}
\author{Ever Ortega}
\title{Informe de pasantía}

\maketitle
\tableofcontents


\clearpage
\section{Objetivos generales}
El objetivo general de la pasantía desarrollada en la empresa NEXO S.A.,
en el área de Informática – Desarrollo de Software, fue aplicar de
manera práctica los conocimientos teóricos adquiridos durante
la formación académica, contribuyendo activamente al mantenimiento,
mejora y optimización de los sistemas informáticos de la organización.
La experiencia tuvo como propósito fortalecer las competencias técnicas,
analíticas y profesionales necesarias para el desempeño eficiente dentro
del ámbito laboral del desarrollo tecnológico.

Asimismo, la pasantía buscó integrar al pasante en un entorno de
trabajo real, fomentando la participación en procesos de programación,
gestión de bases de datos y desarrollo de interfaces,
con el fin de adquirir una comprensión más amplia del ciclo completo de
desarrollo de software dentro de una empresa de servicios financieros.

A través de esta experiencia, se pretendió también promover el
desarrollo de valores y actitudes fundamentales como la responsabilidad,
el compromiso, la ética profesional,
la capacidad de adaptación y el trabajo en equipo,
todos ellos esenciales para una adecuada inserción en el ámbito
laboral tecnológico.

En síntesis, la pasantía tuvo como finalidad consolidar el aprendizaje
teórico-práctico del pasante, generar experiencia profesional en un
entorno real de trabajo y fortalecer la formación integral necesaria
para un desempeño competente,
ético y responsable en el campo de la informática y el desarrollo de software.


\clearpage
\section{Objetivos específicos}

\begin{itemize}
	\ii Aplicar conocimientos de desarrollo de software en entornos de trabajo reales, utilizando herramientas y lenguajes actuales.
	\ii Fortalecer el manejo de bases de datos relacionales mediante la elaboración de consultas SQL (queries) y cursores para el procesamiento eficiente de información.
	\ii Utilizar Python y sus librerías en la creación y mantenimiento de interfaces gráficas de usuario (GUI).
	\ii Integrar buenas prácticas de documentación, control de versiones y seguridad en los procesos de desarrollo.
	\ii Desarrollar competencias en trabajo en equipo, comunicación profesional y resolución de problemas técnicos.
	\ii Reforzar valores de responsabilidad, ética, compromiso y adaptabilidad en el ámbito laboral.
	\ii Aplicar conocimientos técnicos en desarrollo de software demostrando responsabilidad, compromiso y ética profesional.
	\ii Fomentar la proactividad y el trabajo colaborativo en la resolución de problemas del área de desarrollo.
	\ii Desarrollar autonomía y disciplina en la ejecución de tareas relacionadas con programación y manejo de bases de datos.
	\ii Integrar valores de respeto, puntualidad y responsabilidad en el cumplimiento de los objetivos laborales.
	\ii Promover una actitud analítica y crítica orientada a la mejora continua de los sistemas informáticos.
\end{itemize}


\clearpage
\section{Metas}
Durante el desarrollo de la pasantía se alcanzaron las siguientes metas:

\begin{itemize}
	\ii Consolidar los conocimientos teóricos adquiridos en el colegio mediante la práctica en entornos reales de desarrollo.
	\ii Desarrollar independencia técnica, adquiriendo confianza para resolver problemas de programación y optimización de bases de datos.
	\ii Comprender la estructura y dinámica del área de desarrollo dentro de una empresa de servicios financieros y de cobranza.
	\ii Fortalecer habilidades analíticas, orientadas a la mejora de procesos informáticos y la eficiencia de sistemas internos.
	\ii Fomentar la disciplina y la organización, gestionando de manera efectiva los tiempos y prioridades de trabajo.
	\ii El cumplimiento de estas metas tuvo un impacto directo en mi formación profesional, aportando una visión más práctica del desarrollo de software en entornos corporativos y consolidando mi orientación hacia la informática.
\end{itemize}


\clearpage
\section{Descripción de actividades}

Durante las 240 horas de pasantía, se desarrollaron las siguientes actividades técnicas y formativas:

\begin{enumerate}
	\ii Mantenimiento y optimización de bases de datos

	Elaboración y depuración de consultas SQL (queries) para la obtención y análisis de datos requeridos por distintos departamentos de la empresa.
	Creación y manejo de cursores para el procesamiento automatizado de información en volúmenes considerables.
	Verificación de la integridad de datos y análisis de rendimiento de las consultas.
	Aplicación de criterios de normalización y buenas prácticas en la gestión de bases de datos.
	\ii Desarrollo con Python

	Implementación de scripts en Python orientados a la automatización de procesos internos.
	Desarrollo de interfaces gráficas de usuario (GUI) utilizando la librería Tkinter, con el fin de mejorar la usabilidad de herramientas internas.
	Pruebas de funcionamiento y depuración de código, garantizando la estabilidad de los programas desarrollados.
	Integración de módulos y librerías adicionales según las necesidades de los proyectos asignados.

	\ii Apoyo en tareas del área de desarrollo
	Colaboración en la documentación técnica de procesos y procedimientos de los sistemas utilizados.
	Revisión de incidencias y apoyo en la solución de errores.
\end{enumerate}

\clearpage
\section{Herramientas y tecnologías utilizadas}

\begin{itemize}
	\ii Lenguaje de programación: Python.
	\ii Librerías: Tkinter, os, Selenium entre otras.
	\ii Base de datos: PL/SQL.
	\ii Control de versiones: Git.
	\ii Entorno de desarrollo: NeoVim y PL/SQL Developer.
\end{itemize}


\clearpage
\section{Conocimientos y habilidades desarrolladas}

\begin{itemize}
	\ii Programación estructurada y orientada a objetos en Python.
	\ii Análisis y diseño de consultas SQL eficientes.
	\ii Uso de cursores y procedimientos almacenados.
	\ii Diseño de interfaces gráficas con Tkinter.
	\ii Capacidad para documentar procesos técnicos y comunicar avances.
	\ii Fortalecimiento del pensamiento lógico y la resolución de problemas.
\end{itemize}

\clearpage
\section{Conclusiones}
La pasantía en la empresa NEXO S.A. constituyó una experiencia sumamente
enriquecedora desde el punto de vista técnico y profesional.
La posibilidad de trabajar en proyectos reales dentro del área de
desarrollo permitió consolidar los conocimientos adquiridos durante la carrera,
ampliando las competencias en programación,
manejo de bases de datos y desarrollo de interfaces.

Asimismo, la interacción con el equipo técnico facilitó la comprensión
del flujo de trabajo en un entorno empresarial,
reforzando habilidades de comunicación, responsabilidad y colaboración.

El desarrollo de tareas utilizando Python,
Tkinter y SQL permitió fortalecer la lógica de programación,
el análisis de datos y la capacidad de diseñar
soluciones informáticas eficientes.

En conclusión, la pasantía representó una etapa fundamental en mi
formación profesional, aportando experiencia práctica y reafirmando
mi compromiso con la mejora continua y la excelencia en el ámbito
del desarrollo de software.

\clearpage
\section{Cuadro de aprobación}
 {
  \renewcommand{\arraystretch}{2}
  \begin{center}
	  \begin{tabular}{|c|l|}
		  \hline

		  \multirow{4}{*}{PRESENTADO POR:} & NOMBRE Y APELLIDO DEL PASANTE:                              \\ \cline{2-2}
		                                   & \                                                           \\ \cline{2-2}
		                                   & FIRMA:                                                      \\ \cline{2-2}
		                                   & \                                                           \\
		  \hline

		  \multirow{8}{*}{APROBADO POR:}   & NOMBRE Y APELLIDO DEL TUTOR DE PASANTÍA:                    \\ \cline{2-2}
		                                   & \                                                           \\ \cline{2-2}
		                                   & FIRMA:                                                      \\ \cline{2-2}
		                                   & \                                                           \\ \cline{2-2}
		                                   & Nombre y apellido del Director de la institución educativa: \\ \cline{2-2}
		                                   & \                                                           \\ \cline{2-2}
		                                   & FIRMA:                                                      \\ \cline{2-2}
		                                   & \                                                           \\ \hline

		  LUGAR Y FECHA:                   & Asunción, \today                                            \\
		  \hline
	  \end{tabular}
  \end{center}
 }
\end{document}
